\documentclass[a4paper,11pt]{article}

%A Few Useful Packages
\usepackage{marvosym}
%\usepackage{fontspec} 					%for loading fonts
\usepackage{xunicode,xltxtra,url,parskip} 	%other packages for formatting
\RequirePackage{color,graphicx}
\usepackage[usenames,dvipsnames]{xcolor}
%\usepackage{fullpage}

%\usepackage[big]{layaureo} 				%better formatting of the A4 page
\usepackage[bottom=0.5cm,top=0.7cm,right=1.5cm,left=1.5cm]{geometry}
% an alternative to Layaureo can be ** \usepackage{fullpage} **
\usepackage{supertabular} 				%for Grades
\usepackage{titlesec}					%custom \section
\usepackage{graphicx}
\graphicspath{ {images/} }

%Setup hyperref package, and colours for links
\usepackage{hyperref}
\definecolor{linkcolour}{rgb}{0,0.2,0.6}
\hypersetup{colorlinks,breaklinks,urlcolor=linkcolour, linkcolor=linkcolour}

%FONTS
\defaultfontfeatures{Mapping=tex-text}
%\setmainfont[SmallCapsFont = Fontin SmallCaps]{Fontin}
%%% modified for Karol Kozioł for ShareLaTeX use
%\setmainfont[
%SmallCapsFont = Fontin-SmallCaps.otf,
%BoldFont = Fontin-Bold.otf,
%ItalicFont = Fontin-Italic.otf
%]
%{Fontin.otf}
%%%
\usepackage{helvet}
%CV Sections inspired by: 
%http://stefano.italians.nl/archives/26
\titleformat{\section}{\Large\scshape\raggedright}{}{0em}{}[\titlerule]
\titlespacing\section{0em}{0.5em}{0.25em}

%Tweak a bit the top margin
%\addtolength{\voffset}{-1.8cm}

%Italian hyphenation for the word: ''corporations''
\hyphenation{im-pre-se}

%-------------WATERMARK TEST [**not part of a CV**]---------------
\usepackage[absolute]{textpos}

\setlength{\TPHorizModule}{35mm}
\setlength{\TPVertModule}{\TPHorizModule}
\textblockorigin{2mm}{0.65\paperheight}
\setlength{\parindent}{0pt}

%--------------------BEGIN DOCUMENT----------------------
\begin{document}

%WATERMARK TEST [**not part of a CV**]---------------
%\font\wm=''Baskerville:color=787878'' at 8pt
%\font\wmweb=''Baskerville:color=FF1493'' at 8pt
%{\wm 
%	\begin{textblock}{1}(0,0)
%		\rotatebox{-90}{\parbox{500mm}{
%			Typeset by Alessandro Plasmati with \XeTeX\  \today\ for 
%			{\wmweb \href{http://www.aleplasmati.comuv.com}{aleplasmati.comuv.com}}
%		}
%	}
%	\end{textblock}
%}

\pagestyle{empty} % non-numbered pages

\font\fb=''[cmr10]'' %for use with \LaTeX command

%--------------------TITLE-------------
\par{\centering
		{\Huge  \textsc{Guillem Duran Ballester}
	}\bigskip\par}

\par{\centering{	
\begin{tabular}{p{2.5cm}cccccc}
    &\textsc{Phone:}     & +34 662 093000 &
    \textsc{email:}     & \href{mailto:guillem.db@gmail.com}{guillem.db@gmail.com} &
    \textsc{Github:}     & \href{https://github.com/Guillem-db}{guillem-db}\\
\end{tabular}
}}

%\centering{

%\begin{tabular}{lllll}\n
%\toprule 
% \n\textbf{Phone} & \textbf{Email} & \textbf{Github} \\\\\n
%\midrule\n   +34 662 09300 &  \href{mailto:guillem.db@gmail.com}{guillem.db@gmail.com} & %\href{https://github.com/Guillem-db}{guillem-db}\\\n

%\bottomrule\n
%\end{tabular}\n
%\end{center}
%}

%--------------------SECTIONS-----------------------------------
%Section: Personal Data
%\resizebox{\textwidth}{!}



%\begin{tabular}{rl}
%    \textsc{Place and Date of Birth:} & Palma de Mallorca, Spain  | 16 February 1990 \\
%    \textsc{Address:}   & Emili Darder Batle 12 6c, 07013,Palma, Balears, Spain \\
%    \textsc{Phone:}     & +34 662 09300\\
%    \textsc{email:}     & \href{mailto:guillem.db@gmail.com}{guillem.db@gmail.com}\\
%   \textsc{Github:}     & \href{https://github.com/Guillem-db}{guillem-db}
%\end{tabular}




%Section: Work Experience at the top
\section{Accomplishments}

%\begin{tabular}{l|p{11cm}}
\begin{tabular}{cc}
    \begin{minipage}{.5\linewidth}
\begin{tabular}{p{1.5cm}|p{5.5cm}}
\multicolumn{2}{c}{\Large Education \& Papers} \\\multicolumn{2}{c}{}\\

 \textsc{Jun 2018} & \textbf{Solving Atari Games Using Fractals And Entropy, \href{https://arxiv.org/abs/1807.01081}{\emph{Arxiv.org (In revision)} }}
 \\\multicolumn{2}{c}{}\\

\textsc{Aug.}  & \textbf{Happiness, an inside job}\\2017&\href{https://dl.acm.org/citation.cfm?id=3110132}{\emph{ASONAM 2017}}\\&\footnotesize{\textit{Churn prediction using employee likability, engagement and relative happiness}}\\\multicolumn{2}{c}{}\\

 \textsc{May 2017} & \textbf{General Algorithmic Search, \href{https://arxiv.org/abs/1705.08691}{\emph{Arxiv.org} }}
 \\\multicolumn{2}{c}{}\\

\textsc{Oct.} 2016 & Degree Telecom. engineering: spec. in \href{https://eetac.upc.edu/en/study/bachelors-deegrees/telematics-engineering}{\textsc{Telematics}}, \textbf{Politechnic Univ. of Catalonia}, Spain\\&\footnotesize{\textit{Thesis: 'Time series and graph analysis in the Jupyter notebook'}}\\

 \multicolumn{2}{c}{}\\\multicolumn{2}{c}{}\\\multicolumn{2}{c}{\Large Work experience} \\\multicolumn{2}{c}{}\\

\textsc{May 2017 - Present} & \href{http://faculty.uaeu.ac.ae/jose/}{ \textbf{Research assistant}} at \textsc{University of the United Arab Emirates} (Al Ain - Abu Dhabi), UAE.  \emph{Robots and media lab}\multicolumn{2}{c}{}\\

\textsc{Jan - May 2017} & \textbf{Freelance data scientist}, Palma de Mallorca, Spain\multicolumn{2}{c}{}\\

\textsc{Jul - Dec 2017} & \href{http://faculty.uaeu.ac.ae/jose/}{ \textbf{AI engineer}} at \textsc{Source\{d\}} Madrid, Spain.  \emph{Reinforcement learning research}\multicolumn{2}{c}{}\\


\textsc{Dec 2016 - Present} & \href{http://cio.umh.es/en/investigacion/investigadores/jmamigo}{\textbf{Team member}} at \emph{Aplicaciones de los Sistemas Dinámicos Discretos y Continuos, MTM2016-74921-P (AEI/FEDER, UE)}. \textsc{UMH Univ.}, Elche, Spain \multicolumn{2}{c}{}\\

 \textsc{Jan - Dec 2016} & \textbf{AI researcher} at \textsc{HCSoft Programación S.L}, Murcia, Spain.  \emph{Metaheuristic and AI research}\multicolumn{2}{c}{}\\

\textsc{Jun-Dec 2014} & \textbf{Cloud computing DevOps internship} at \textsc{UPC}. \footnotesize{Installed, configured and administrated an OpenNebula-based cloud computing.}\\\multicolumn{2}{c}{}\\

\multicolumn{2}{c}{}
\end{tabular}
\end{minipage} &
\begin{minipage}{.5\linewidth}

%Section: Scholarships and additional info
%\section{Talks and lectures}

\begin{tabular}{p{1cm}|p{7cm}}
\multicolumn{2}{c}{}\\\multicolumn{2}{c}{}\\\multicolumn{2}{c}{\Large Talks} \\\multicolumn{2}{c}{}\\

\textsc{Oct.}  & \textbf{\href{https://2018.es.pycon.org/}{Talk \& Workshop PyCon ES (Accepted)}}\\2018&\emph{Talk: Hacking RL (Spanish version); Workshop: Introduction to data science Workshop}\\&\\\multicolumn{2}{c}{}\\

\textsc{Jul.}  & \textbf{\href{https://ep2018.europython.eu/conference/talks/hacking-reinforcement-learning}{Talk at EuroPython 2018}}\\2018&\emph{Hacking Reinforcement Learning}\\&\footnotesize{\textit{Presenting planing algorithms derived from Fractal AI theory for playing Atari at a superhuman level.}}\\\multicolumn{2}{c}{}\\


\textsc{Nov.}  & \textbf{\href{https://twitter.com/Miau_DB/status/926728441247993856}{Invited speaker to PiterPy 2017 conferences}}\\2017&\emph{Reinforcement learning for developers}\\&\footnotesize{\textit{An introduction to RL with Rick \& Morty }}\\\multicolumn{2}{c}{}\\


\textsc{Jul.}  & \textbf{\href{https://www.youtube.com/watch?time_continue=1\&v=JUErYqjf5Zw}{Talk at EuroPython 2017} }\\2017&\emph{Inside Airbnb: Visualizing data that includes geographic locations}\\&\\\multicolumn{2}{c}{}\\

\textsc{May.}  & \textbf{\href{https://pydata.org/barcelona2017/schedule/presentation/19/}{Talk at PyData Barcelona}}\\2017&\emph{Happiness inside a job: a social network analysis.}\\& \footnotesize{\textit{A talk about how to apply social network analysis techniques to machine learning.}}\\\multicolumn{2}{c}{}\\

\textsc{Mar.}  & \textbf{Lecture at Universidad de Zaragoza}\\2017&\emph{Introduction to Fractal AI theory for researchers and Phd. students. (10 Hours)}\\&\footnotesize{\textit{Introductory course to fractal AI methods for solving complex artificial intelligence environments.}}\\\multicolumn{2}{c}{}\\

\textsc{Feb.} & \textbf{\href{https://github.com/PyDataMallorca/WS_Introduction_to_data_science}{Workshop at PyData Mallorca}}\\2017&\emph{Introduction to data science (5 Hours)}\\\multicolumn{2}{c}{}\\

\textsc{oct.}  & \textbf{\href{https://github.com/Guillem-db/PyconEs-2016/blob/master/PSAD\%20Cosmic\%20Billiards.ipynb}{Talk at PyconEs 2016}}\\2016&\emph{Per Shaolin ad astra.}\\&\footnotesize{ \textit{How to calculate and visualize the trajectory of the Juno spacecraft using Shaolin.}}\\\multicolumn{2}{c}{}\\

\textsc{Jul.} & \textbf{\href{https://ep2016.europython.eu/conference/talks/interactive-data-kung-fu-with-shaolin}{Hot topic talk at EuroPython 2016}}\\2016&\emph{Interactive Data Kung Fu with \href{https://github.com/HCsoft-RD/shaolin/tree/master/examples}{Shaolin}.}\\&\footnotesize{\textit{Interactive data visualization using Shaolin, a python library I programmed}}
\\\multicolumn{2}{c}{}\\\multicolumn{2}{c}{}
\\\multicolumn{2}{c}{}\\
\end{tabular}

\end{minipage}

%\end{minipage}
\end{tabular}





\section{AI Research experience}

Over the years, I have managed to develop Fractal AI, a research project consisting on a theory of artificial intelligence based on first principles. I am part of a two people team that managed to develop a concept into a working theory, and developed a new family of planning algorithms that greatly outperform any other existing alternative.

\textbf{As an undergraduate} I researched on artificial intelligence as a hobby. During this time, I helped develop the theoretical foundations of our theory, and helped design and test a new set of environments to be solved by our agents.

At \textbf{HCSoft}, I developed and tested our novel AI theory, and learned how to find creative solutions to overcome technical limitations. My work there involved writing a Python implementation of our algorithms to be used with OpenAI gym, and as a global optimization algorithm. I also was in charge of designing and writing visualization and debugging tools.

At, \textbf{Source\{d\}}, I applied our techniques to leverage Reinforcement Learning algorithms. I learned how to communicate and work with a team, and follow good software development practices.

At, \textbf{UAE University}, I developed tools to generate high quality Reinforcement Learning datasets ready to use for research, and scaled our algorithms to be used in an Amazon Web services cluster.

We have published our findings as an \href{https://github.com/FragileTheory/FractalAI}{open source project}, and have documented our research in a \href{http://entropicai.blogspot.com.es/2017/06/openai-first-record.html}{blog} and a \href{https://www.youtube.com/user/finaysergio/videos}{YouTube} channel:\\


\begin{tabular}{ccc}
 \begin{minipage}{.30\linewidth}
        \large \textbf{Skills developed}
        \normalsize
        \begin{itemize}
          \item I'm an very fast learner
          \item I'm highly motivated
          \item Knowledge sharing
          \item Understanding AI papers
          \item Proposing new approaches to solve existing problems
          
        \end{itemize}
    \end{minipage} & 
    \begin{minipage}{.28\linewidth}
        \large \textbf{Python tools I used}
        \normalsize
        \begin{itemize}
          \item Git, Pycharm, Jupyter
          \item Pandas, numpy
          \item Keras, TensorFlow
          \item Scikit-learn, Scipy
          \item OpenAI, Mujoco 
          \item Mpl, Bokeh, Plotly
          
        \end{itemize}
    \end{minipage} & 
    \begin{minipage}{.24\linewidth}
        \large \textbf{I researched} 
        \normalsize
        \begin{itemize}
          \item \href{https://youtu.be/fq_uxGyuVhU?t=10m44s}{Atari Games}
          \item \href{https://youtu.be/jpXq-NCg1-E}{Sega Games}
          \item \href{https://www.youtube.com/watch?v=kSyae8URr54&t=79s}{Optimization}
          \item \href{https://www.youtube.com/watch?v=J9kW1lhT06A}{Multi-agent flight}
          \item \href{https://www.youtube.com/watch?v=R61FRUf-F6M}{Path finding}
          \item \href{https://youtu.be/HLbThk624jI?t=39s}{Low prob. sampling}
         
        \end{itemize}
    \end{minipage}
\end{tabular}




\normalsize




\section{Available on GitHub}
\begin{tabular}{cc}
 \begin{minipage}{.5\linewidth}


   
    
        
        \normalsize
        \begin{itemize}
          \item \href{https://github.com/Guillem-db/PyconEs-2016/blob/master/PSAD\%20Cosmic\%20Billiards.ipynb}{Physics}: \emph{Calculating the trajectory of an spacecraft.}\\
          \item \href{https://github.com/Guillem-db/graphs-and-network-analysis/blob/master/1\%20-\%20Geant\%20network\%20and\%20introduction\%20to\%20graph\%20theory.ipynb}{Networks}: \emph{Analyzing the GEANT network using graph theory.}\\
          
          \item \href{https://github.com/Guillem-db/lectures-on-AI/blob/master/1\%20-\%20Introduction\%20to\%20AI.ipynb}{Artificial intelligence}: \emph{Introduction to reinforcement learning.}\\
          
          \item \href{https://github.com/Guillem-db/NLP/blob/master/Assignment\%201.ipynb}{NLP}: \emph{Analyzing a corpus using Natural Language Processing.}\\
          
          \item \href{https://github.com/HCsoft-RD/Optimization}{Optimization}: \emph{Notebooks on benchmarking global optimization algorithms.}\\
          
          \item \href{https://github.com/Guillem-db/Seguridad-de-redes}{Hacking}: \emph{How to build a JavaScript-based port scanner.}\\
          
          \item \href{https://github.com/Guillem-db/Smart-cities/blob/master/SmartCities.ipynb}{Teaching}: \emph{A tutorial on smart cities network design.}\\
          
          \item \href{https://github.com/BielStela/InsideAirbnb/tree/master/notebooks}{Data analysis}: \emph{Analyzing data from Airbnb.}        
         
        \end{itemize}
\end{minipage}&

\begin{minipage}{.45\linewidth}


\begin{itemize}
          \item \href{https://github.com/HCsoft-RD/shaolin/blob/master/examples/GraphPlot.ipynb}{Financial markets}: \emph{Analyzing correlation matrices of Forex market data.}\\
          \item \href{https://github.com/BielStela/InsideAirbnb/blob/master/notebooks/Interactive\%20Map.ipynb}{Plotting data on maps}: \emph{Interactive maps displaying data from Airbnb.}\\
          \item \href{https://github.com/HCsoft-RD/shaolin/blob/master/examples/Scatter\%20Plot\%20introduction.ipynb}{Interactive plots}: \emph{Dashboards used to visualize interactive scatter plots.}\\
          \item \href{https://github.com/Guillem-db/PyData-Bcn-2017}{Social networks analysis}: \emph{Data wrangling to build machine learning features using network theory.}\\
          
          \item \href{https://github.com/Guillem-db/docker-base}{Docker container}: \emph{Bleeding edge TensorFlow compiled for Python 3.7.}\\
          
          \end{itemize}
        \Large \underline{Other Activities}\\
        \normalsize
        \begin{itemize}
          \item \textbf{PyData Mallorca organizer} since Feb 2017.\\
          \item \textbf{I love teaching science and Python} in workshops, lectures, and as a private teacher.\\
          
          
        \end{itemize}
        
   
    \end{minipage}
\end{tabular}

    





\end{document}

\begin{minipage}{.76\linewidth}
\includegraphics[width=15cm, height=5cm]{data_skills}
\end{minipage}\\

%Section: Languages
\section{Mas cosas}
\begin{tabular}{rl}
\textsc{English:}&Working proficiency\\
\textsc{Spanish:}&Native\\
\textsc{Catalan:}&Native\\
\end{tabular}

%Section: Education
\section{Education}
\begin{tabular}{rl}

\textsc{Oct.} 2016 & Degree Telecom. engineering: spec. in \textsc{Telematics}, \textbf{Politechnic Univ. of Catalonia}\\&\footnotesize{\textit{Thesis: 'Time series and graph analysis in the Jupyter notebook'}}\\
\end{tabular}


%Section: Languages
\section{Languages}
\begin{tabular}{rl}
\textsc{English:}&Working proficiency\\
\textsc{Spanish:}&Native\\
\textsc{Catalan:}&Native\\
\end{tabular}

\section{Science-related skillset}

\includegraphics[width=17cm, height=5cm]{personal_skills}
\includegraphics[width=17cm, height=8cm]{data_skills}



\textsc{Feb 2017 - Present} & \textbf{Research consultant} at \textsc{Universidad Miguel Hernández}, Elche \\&\emph{Grupo de excelencia investigadora on operations research and complex systems analysis}\\\multicolumn{2}{c}{}\\

\textsc{Feb 2017 - Present} & \textbf{PyData Mallorca organiser}, Palma de Mallorca \\\\\multicolumn{2}{c}{}\\

\section{\\Tools}
\begin{tabular}{ccc}
 \begin{minipage}{.30\linewidth}
        \large Data analysis\\
        \normalsize
        \begin{itemize}
          \item Numpy
          \item Pandas
          \item Scipy
          \item TensorFlow
          \item Scikit-learn
          \item Keras
          \item Openai
        \end{itemize}
    \end{minipage} & 
    \begin{minipage}{.28\linewidth}
        \large Visualization\\
        \normalsize
        \begin{itemize}
          \item Bokeh
          \item Matplotlib
          \item Plotly
          \item GGplot
          \item Holowievs
          \item Datashader
          \item Geoviews
        \end{itemize}
    \end{minipage} & 
    \begin{minipage}{.30\linewidth}
        \large Other\\
        \normalsize
        \begin{itemize}
          \item Linux (Ubuntu \& Kali)
          \item IDEs: Jupyter, Eclipse, Spyder
          \item Excel
          \item Git
          \item SPSS
          \item Tableau
          \item Amazon EC2
        \end{itemize}
    \end{minipage}
\end{tabular}

\multicolumn{2}{c}{\Large Education} \\\multicolumn{2}{c}{}\\
 \textsc{Oct. 2016} & \textbf{MSEE}, \emph{Polytechnic Univ. of Catalonia}\\&\footnotesize{\textit{Thesis: 'Time series and graph analysis in the Jupyter notebook'}}\\\multicolumn{2}{c}{}\\
 
 
 \Large{\underline{Artificial intelligence Research}}\\
\normalsize

I have spend the last four years \href{http://entropicai.blogspot.com.es/2016/02/folding-proteins-with-fractal.html}{collaborating} in the development of a new mathematical framework for solving operations research and artificial intelligence problems. This approach based on \href{http://www.alexwg.org/publications/PhysRevLett_110-168702.pdf}{causal entropic forces} allows us to solve \href{https://www.youtube.com/watch?v=HLbThk624jI}{control theory} problems on \href{https://www.youtube.com/watch?v=4uP-kcqcM1g}{stochastic environments}, \href{https://www.youtube.com/watch?v=J9kW1lhT06A}{multi-agent} environments, and \href{https://www.youtube.com/user/finaysergio/videos?view=0&sort=dd&shelf_id=0}{many other} tasks.\\
I have also worked in problems such as \href{https://www.youtube.com/watch?v=AoiGseO7g1I&t=6s}{path finding}, the unit commitment problem (UCP), and numerical simulations.\\\


\begin{tabular}{p{5cm}p{12cm}}
  \begin{minipage}{\linewidth}
\includegraphics[width=4.25cm, height=3.5cm]{radar_no_bound}
\end{minipage} &
\begin{minipage}{\linewidth}
    
\par{
I am a young data scientist with strong background in Python programming and a highly motivated proactive learner. My job as a data scientist provided me with wide experience in Financial markets, Network analysis, Optimization, and Artificial Intelligence, as well as proficient Python coding skills. If you can think of a problem, I will not stop until coding a solution.
}

\end{minipage} 
\end{tabular}